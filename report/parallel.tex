\subsection{Parallel bot}
\label{parallel}

The approach for the parallel bot is to try to use multiple cores on systems to try to achieve better results in evaluating moves within the time constraints.

The commonly used Python implementation CPython has a global interpreter lock which prevents multiple threads to really run in parallel when the machine provides more than one thread/core\cite{pygil}. To circumvent this limitation Python provides a module to distribute computation over several processes, providing shared memory and other means to comunicate.

The parallel bot can use different heuristics. Its first step is always to evaluate the boards resulting from the possible moves it can do. So in case of a timeout it will always have a move to perform.
The outcomes of the board are sorted and new processes are then spawned to evaluate the future scenarios deriving from the best immediate moves.

\begin{figure}
\begin{algorithmic}
\Function{Max-Prob}{board, player-id}
\State Moves = \textsc{All-possible-Moves}
\State P = Max\{WinProb(m, player-id) $| \; c \in$ Moves\}
\State $C_{max}$ = \{$m \in$ Moves $|$ WinProb(m, player-id) = P\}
\State \Return rand($C_{max}$)
\EndFunction
\end{algorithmic}
\caption{parallel algorithm}
\end{figure}