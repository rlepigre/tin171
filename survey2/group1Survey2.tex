\documentclass[a4paper,11pt]{article}
\usepackage{fancyhdr}
\setlength{\headheight}{11pt}
\pagestyle{fancyplain}
%\renewcommand{\chaptermark}[1]{\markboth{#1}{}}

\lhead{ }
\rhead{}
%\renewcommand{\headrulewidth}{0.0pt}

\lfoot{Group X: Y survey}
%COMMENT: Y is the topic which has been assigned to you
\cfoot{\thepage}
\rfoot{}

%
%%    homebrew commands -- to save typing
\newcommand\etc{\textsl{etc}}
\newcommand\eg{\textsl{eg.}\ }
\newcommand\etal{\textsl{et al.}}
\newcommand\Quote[1]{\lq\textsl{#1}\rq}
\newcommand\fr[2]{{\textstyle\frac{#1}{#2}}}
\newcommand\miktex{\textsl{MikTeX}}
\newcommand\comp{\textsl{The Companion}}
\newcommand\nss{\textsl{Not so Short}}


\begin{document}
%-----------------------------------------------------------
\title{
(probably) The same title\\
you used \\
for your pre-proposal\\}
\author{Group X: Author 1, Author 2, Author 3, Author 4}
\maketitle
%-----------------------------------------------------------
\begin{abstract}\centering
%%COMMENT
no more than 10 lines!
\end{abstract}

\section{Introduction}
\subsection{The problem or area you are interested in}
(extract an accurate description of the problem from the project
description or from your pre-proposal, or lift that material to a general problem specification)

\subsection{Significance}
(Why is the problem interesting or significant? What will a solution achieve?)

\subsection{Aspects of the problem covered by this survey}
(describe these briefly. You could write a small paragraph for each of them, highlighting keywords, and citing the relevant papers. The keywords could be used as sub-section headings in the survey proper, in section 3.)

\subsection{Important aspects of the problem not covered here}
(Aspects of the problem one must know to understand it completely, but not included in the discussion. Use this subsection to ensure a good coverage of the relevant literature without going through much detail)

\section{Framework}
(Here you describe which solution approach(es) you would have chosen if
this would have been your course project. Use your own words, write down
definitions, etc., )
\subsection{Define formally an instance of the problem}
(Either mathematically, or using pseudo-code. Also, give an example)
\subsection{Benchmarks}
(How would you measure the performance of a program that solves the problem?
Find instances/benchmarks, or make your own).

\section{Survey}
\subsection{Results from the literature}
(describe briefly the scientific papers you found relevant to the problem.  Ideally, a commented summary of each paper, like an extended annotation.  Cite references. You could organize this section by keyword, one sub-section for each keyword you specified in 1.3).

\subsection{Tools and programs available for the problem, or for closely related ones}
(describe these briefly. Say how they can be used, and how a solution to your problem can build on them, or differ from them).

\section{References}

For instructions on how to handle references, please see the template for
the project proposal.


\section{Appendix}
The previous sections (excluding title page and references) are limited to 6 pages. Here, you can put in anything that didn't fit in there, but that you still think is essential.
\end{document}
